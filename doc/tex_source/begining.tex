\section{Bevezetés}

A beszédszintézis célja, hogy beszélő emberi hangot hozzon létre mesterségesen. A gépekkel való emberi kommunikáció elengedhetetlen része. Napjainkban fontos felhasználási területe a vakok és gyengénlátók számára készült kisegítő lehetőségek gyártása.

\subsection{Főbb szintézis technológiák}
\subsubsection{Összefűzéses szintézis}
Ebben a módszerben előre felvett szegmenseket fűznek össze. A megfelelő szöveg(text)-darabokat megfelelő felvett hangdarabokkal párosítják, és ezeket egymás után mondatják ki a programmal.
\subsubsection{Artikulációs szintézis}
Ez a módszer alapvetően az emberi hangképzést hivatott lemodellezni.
\subsubsection{HMM-alapú szintézis}
A rejtett Markov-modell alapú szintézis lényege, hogy statisztikai alapon próbálják modellezni a beszéd különböző paramétereit, és minden kombinációra becsülik a helyességének valószínűségét. A legnagyobb valószínűségűt választva kapható meg a kívánt hang.