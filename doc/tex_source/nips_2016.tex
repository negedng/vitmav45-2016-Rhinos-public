\documentclass{article}

% if you need to pass options to natbib, use, e.g.:
% \PassOptionsToPackage{numbers, compress}{natbib}
% before loading nips_2016
%
% to avoid loading the natbib package, add option nonatbib:
% \usepackage[nonatbib]{nips_2016}
\usepackage[final]{nips_2016} % produce camera-ready copy

% to compile a camera-ready version, add the [final] option, e.g.:
% \usepackage[final]{nips_2016}

\usepackage[utf8]{inputenc} % allow utf-8 input
\usepackage[T1]{fontenc}    % use 8-bit T1 fonts
\usepackage{hyperref}       % hyperlinks
\usepackage{url}            % simple URL typesetting
\usepackage{booktabs}       % professional-quality tables
\usepackage{amsfonts}       % blackboard math symbols
\usepackage{nicefrac}       % compact symbols for 1/2, etc.
\usepackage{microtype}      % microtypography

\title{Deep Neural Network based Text-to-Speech
	\\
	Beszédszintézis Mély Neurális Hálóval}

% The \author macro works with any number of authors. There are two
% commands used to separate the names and addresses of multiple
% authors: \And and \AND.
%
% Using \And between authors leaves it to LaTeX to determine where to
% break the lines. Using \AND forces a line break at that point. So,
% if LaTeX puts 3 of 4 authors names on the first line, and the last
% on the second line, try using \AND instead of \And before the third
% author name.

\author{
	Tamas Szanto\\
	\texttt{tmas.szanto@gmail.com} \\
	\AND
	Gergely D. Nemeth\\
	\texttt{NeGeD.NG@gmail.com} \\
}

\begin{document}
% \nipsfinalcopy is no longer used

\maketitle
\chapter{}
\begin{abstract}
  Human-like communication with the computers is a major application of Computer Science. The Text-to-Speech methods are significant part of the project. This paper is a review of the authors experience about the usefulness of the Deep Neural Networks' new fields regarding to the \textbf{TTS} problem.
\end{abstract}

\begin{abstract}
	A Számítástechinka egyik meghatározó célja a számítógépekkel való emberszerű kommunikáció elérése. A Beszédszintézis ennek elengedhetetlen területe. Jelen dokumentum a szerzők a Mély Neurális Hálók nyújtotta lehetőségek a témában való alkalazásából szerzett tapasztalatainak összefoglalója. 
\end{abstract}

\clearpage

\chapter{}
\section{WaveNet tapasztalatok}

Első célnak a beszédszintézis WaveNet[1] alapú megoldását tűztük ki. Ennek lényege az aktuális időpillanatban a hanghullám értekének az előző, már meghatározott értékek és a szövegfeldolgozásból kapott címkék segítségével történő meghatározása. Két rendszertervet dolgoztunk ki, azzal a különbséggel hogy a fonéma hosszának a becslése egy külön hálóban történik(\ref{wavenet-1}. ábra) vagy a kimenetről van visszavezetve(\ref{wavenet-2}. ábra). Utóbbi esetben a hullámérték mellett lenne egy másik kimeneti érték is, amely azt adná meg hogy kezdődjön-e az új fonéma.

\begin{figure}[h]
	\par
	\centering
	\medskip
	\includegraphics[width=\textwidth,keepaspectratio]{wavenet_0_2}
	\caption{WaveNet külön fonéma hossz becslő hálóval}
	\label{wavenet-1}
\end{figure}
\begin{figure}[h]
	\par\centering\medskip
	\includegraphics[width=\textwidth,keepaspectratio]{wavenet_0_3}
	\caption{WaveNet fonéma hossz becsléssel kiegészítve}
	\label{wavenet-2}
\end{figure}

A megvalósítás során először már meglévő, GitHub-on elérhető implementációkat próbáltunk ki.
\footnote{https://github.com/tomlepaine/fast-wavenet}
\footnote{https://github.com/ibab/tensorflow-wavenet}
\footnote{https://github.com/basveeling/wavenet}
\footnote{https://github.com/usernaamee/keras-wavenet}
Ezek közül az egyiket sikerült egy adott mondatra tanítanunk, azt vissza tudta generálni pontosan[8]. Azonban más mondatokkal való továbbtanítás esetén már nem volt működőképes.

Ezután saját implementációval próbálkoztunk a DeepMind által publikált cikk alapján[1]. Ez az implementáció lényegesen egyszerűbb volt mint a cikkben meghatározott, célunk csak valamilyen kis zörej előállítása volt. Azonban konstans (néma) hangon kívül mást nem sikerült előállítanunk.

A fent említett kísérletek után egyértelművé vált, hogy túl nagy feladatba kezdtünk bele. Ezért visszaléptünk az eredeti célunktól és folytatásképpen a hagyományos DNN alapú beszédszintézissel haladtunk tovább.
\chapter{}
\section{Egyszerű DNN model felépítése}
\label{dnn_model}
A továbbiakban egy DNN model alkalmazása olvasható. A megismert cimkéket továbbiakkal egészítettük ki, majd ez alapján generáltunk gerjesztési és spektrális paramétereket, amikből előállítható az audio.

\includegraphics[width=\textwidth,keepaspectratio]{dnn_struct}

\subsection{Háló modell}
Külön hálón tanítottuk a spektrális és a gerjesztési paramétereket. 

Előrecsatolt mély neurális hálózatot építettünk fel, 6 rejtett réteggel, tanh és sigmoid aktivációs függvényekkel, SGD optimalizálóval és MSE költségfüggvénnyel. A háló rétegeire Dropoutot is használtunk. (/!TODO ref)
A megállást early stoppinggal detektáltuk.

/!TODO háló modell kép
\chapter{}
\section{Bemeneti és kimeneti paraméterek}
\subsection{Szövegfüggő címkék}
A címke generálás egységeként egy mondatot használtunk fel. Ezt az nltk eszköz segítségével elemeztük majd az eredményekből állítottuk össze a címkéket. A címkék alapjául az aktuális fonémák szolgátak, később még ezek lettek tovább bontva idő keret szerint.A fonéma azonosítására az OneHot kódolást alkalmaztuk, azaz minden fonémát 40 értékkel azonosítottunk.

A következő szöveg függő címkéket hoztuk létre:

\begin{itemize}
	\item aktuális fonéma és az őt körülvevő két-két fonéma
	\item hangsúly
	\item megelőző és követő fonémák száma a szóban 
	\item távolság hangsúlyos fonémától mindkét irányban
	\item szó szófaja
	\item szó pozíciója a mondatban
	\item fonémák száma az aktuális szóban és két szomszédjában
	\item szavak száma a mondatba
	\item fonémák száma a mondatban
\end{itemize}

\subsection{Bemeneti paraméterek}
A szövegfüggő fonéma címkéket egészítettük ki a fonémában található időkeretek számával és a keret fonémán belüli elhelyezkedésével. Így a kerethez tartozó 215 bemeneti paraméter az alábbiakból tevődik össze:


\begin{minipage}{0.5\textwidth}
	\begin{itemize}
		\item 0-200 A 40-40 paraméter a kvinfón(2-1-2) fonémáira
		\item 200-213 további szövegfüggő címkék
		\item 213 A fonémán belüli időkeretek száma
		\item 214 A keret elhelyezkedése a fonémán belül
	\end{itemize} 
\end{minipage} \hfill
\begin{minipage}{0.5\textwidth}
	\centering
	Címkék létrehozásának folyamata
	\includegraphics[width=5cm,keepaspectratio]{tag_struct}
\end{minipage} \hfill
\subsection{Kimeneti paraméterek}
Az előző felsorolást folytatva az aktuális időkerethez tartozó elvárt a kimenet paraméterek a következők:

 
\begin{itemize}
	\item 215 a keret hangmagasság értéke
	\item 216-242 a Mel-Cepstrum 26 paramétere
	\item 242 a fonémán belüli keretek száma
\end{itemize}
\subsection{Spektrális és gerjesztési paraméterek}
Mint említettük a prediktálás alapja a spektrális és gerjesztési paraméterek megadása keretenként. Ezen paraméterek segítségével a PyPSTK (/!TODO ref) python csomag használásával állíthatjuk elő az audio kimenetet, valamint hasonlóképpen ezt a csomagot használjuk az adataink a tiszta hangból való előállítására.
\subsubsection{Mel-Cepstrum}
/!TODO

\includegraphics[width=\textwidth,keepaspectratio]{audio_raw}

\includegraphics[width=\textwidth,keepaspectratio]{audio_mc}
\chapter{}
\section{Megvalósítás}
\subsection{Használt erőforrások}
A háló összerakást saját gépen kezdtük el. Majd a véglegesítését és a tanításokat AWS szerveren, NVIDIA Tesla K80 12GB videokártyán végeztük. Jupyter notebook-ban futtatuk a tanításokat, ezek elmentett adatait aztán Tensorboard segítségével elemeztük.

\subsection{Eredmények}
\subsubsection{Zöngésség}
Zöngésség becslésére nagyon jó, alapfrekvencia (pitch érték) meghatározására használhatóan jó és a spektrális paraméter (Mel-Cepstrum) jósolására kezdetleges eredményeket sikerült elérnünk. (utóbbiról ezért nem is mellékeltünk tanítási és validációs grafikonokat). Hiperparaméter optimalizálás tekintetében egyelőre kézzel dolgoztunk (ez még javításra szorul), már így is sikerült használható pontosságot  elérnünk.

\textit{(A grafikonk y tengelyén az MSE hiba értéke x tengelyén az elvégzett epoch-ok száma látható.)}

\begin{minipage}{0.5\textwidth}
	Zöngésség becslése egyszerűbb, futásidő tekintetében gyorsabb feladatnak bizonyult. Ezért itt nem alkalmaztunk early stopping-ot, hanem a több próbálkozás után a fixen 410 epoch-kal történő tanításnál maradtunk.
	
	A teszt adatainkon elért eredmények:
	
	pontosság 0.8342
	
	költségfüggvény: 0.1560
	
	\textit{(MSE hiba értékekkel számítva)}
\end{minipage}
\begin{minipage}{0.5\textwidth}
	\flushright	
	\includegraphics[width=0.8\textwidth,keepaspectratio]{uv_fig}
\end{minipage}
\subsubsection{Alapfrekvencia}
Az alapfrekvencia tanítása esetén alkalmaztunk early stoppig-ot. Ez azonban (a paramétereinek állítása után is) túl hamar állt meg. Ennek következtében több tanítást is elvégeztünk egymás után. Az itt kapott eredményeink messze nem olyan szépek mint a zöngésségé, de ezek is használhatónak bizonyultak.

A teszt adatainkon elért eredmények:

pontosság: 0.4243

költségfüggvény: 0.07214

\textit{(MSE hiba értékekkel számítva)}

\includegraphics[width=\textwidth,keepaspectratio]{pitch_fig}
\subsubsection{MC paraméterek}
MC tanításakor kiütközött, hogy a bemeneti paramétereink nagyon megegyezőek voltak fonémán belül, így darabos lett a tanítás. Ez a magasabb MC értékekre jelentős hibát okozott, így megpróbáltuk kevesebb mc érték használatát pontosítani.

\begin{minipage}{0.5\textwidth}	
	A teszt adatainkon elért eredmények:
	
	pontosság 0.2927
	
	költségfüggvény: 0.2086
	
	\textit{(MSE hiba értékekkel számítva)}
\end{minipage}
\begin{minipage}{0.5\textwidth}
	\flushright	
	\includegraphics[width=\textwidth,keepaspectratio]{length_fig}
\end{minipage}
\subsubsection{Fonéma hossz}
Fonéma hossz becslésnél csoportosítást valósítottunk meg, a fonémákat 8 csoportba osztottuk, a csoportok között 12,5 ms eltéréssel és a csoportra végeztünk classificationt. Erre azért volt szükség mivel viszonylag kevés, géppel annotált tanító adat állt a rendelkezésünkre, ezek alapján nem sikerült használható eredményeket elérni. Az adatokban szereplő eredeti hosszak vizsgálata után ezt a csoportosítási módot választottuk, két szempont a megfelelő differenciálódás és a jelentősen kilógó adatok levágása volt. Tehát minden fonémánk 40 és 140 ms közé került. (A tanító adatainkat is ez alapján módosítottuk még a feldolgozási fázisban, minden keretnél hosszabb fonémát 140 ms-nél levágunk.)

\begin{minipage}{0.5\textwidth}	
	A teszt adatainkon elért eredmények:
	
	pontosság 0.4619
	
	költségfüggvény: 0.02419
	
	\textit{(MSE hiba értékekkel számítva)}
\end{minipage}
\begin{minipage}{0.5\textwidth}
	\flushright	
	\includegraphics[width=\textwidth,keepaspectratio]{length_fig}
\end{minipage}

\subsubsection{Teljes becslő}
A fenti paraméterek ismeretében teljes TTS végezhető. 

A mondat alapján fonéma hosszakat becslünk, majd párhuzamosan zöngésség-alapfrekvencia és mc paraméter becslés folyik, végül a kapott eredményekből előállítható a kimenet.
\chapter{}
\section{További lehetőségek}
\label{future_plans}
Fonéma hossz becslés

További cimkék, pl. frame in sentence paraméter.

Más spektrális paraméterek.

WaveNet további fejlesztése.

\clearpage
\chapter{}
\section*{References}



\small


[1] Fukada, T., Tokuda, K., Kobayashi, T. \ \& S. Imai,\ (1992) {\it An adaptive algorithm for mel-cepstral 
analysis of speech} Proc. ICASSP-92 , pp.137–140, Mar. 1992.

\chapter{}
\subsubsection*{Köszönetnyilvánítás}

A dokumentum és a szerzők munkái nagyban támaszkodnak a Budapesti Műszaki és Gazdaságtudományi Egyetemen tartott Deep Learning a gyakorlatban Python és LUA alapokon tárgy keretei között elhangzott információkra


\end{document}
