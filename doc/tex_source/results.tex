\section{Megvalósítás}
\subsection{Használt erőforrások}
A háló összerakást saját gépen kezdtük el. Majd a véglegesítését és a tanításokat AWS szerveren, NVIDIA Tesla K80 12GB videókártyán végeztük. A tanításokat Jupyter notebook-ban végeztük, ezek elmentett adatait aztán tensorboard segítségével elemeztük.

\subsection{Eredmények}

Zöngésség becslésére nagyon jó, alapfrekvencia (pitch érték) meghatározására használhatóan jó és a spektrélis paraméter (Mel-Cepstrum) jósolására kezdetleges eredményeket sikerült elérnünk. (utóbbiról ezért nem is mellékeltünk tanítási és validációs grafikonokat). Hiperparaméter optimalizálás tekintetében egyelőre kézzel dolgoztunk (ez még javításra szorul), már is sikerült használható pontosságot  elérnünk.

\textit{(A grafikonk y tengelyén az MSE hiba értéke x tengelyén az elvégzett epochok szmáa látható.)}

\begin{minipage}{0.5\textwidth}
Zöngésség becslése egyszerűbb, futásidő tekintetében gyorsabb feladatnak bizonyult. Ezért itt nem alkalmaztunk early stopping-ot, hanem a több próbálkozás után a fixen 410 epoch-kal történő tanításnál maradtunk.

A teszt adatainkon elért eredmények:

pontosság 0.8342

költségfüggvény: 0.1560

\textit{(MSE hiba értékekkel számítva)}
\end{minipage}
\begin{minipage}{0.5\textwidth}
	\flushright	
	\includegraphics[width=0.8\textwidth,keepaspectratio]{uv_fig}
\end{minipage}

Az alapfrekvencia tanítása esetén alkamaztunk early stoppig-ot. Ez azonban (a paramétereinek állítása után is) túl hamar állt meg. Ennek következtében több tanítást is elvégeztünk egymás után. Az itt kapott eredményeink messze nem olyan szépek mint a zöngésségé, de ezek is hasznaálhatónak bizonyultak.

A teszt adatainkon elért eredmények:

pontosság 0.07214

költségfüggvény: 0.4243

\textit{(MSE hiba értékekkel számítva)}

\includegraphics[width=\textwidth,keepaspectratio]{pitch_fig}